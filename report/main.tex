\documentclass[11pt]{article}

\usepackage{sectsty}
\usepackage{graphicx}
\usepackage{amsmath}

% Margins
\topmargin=-0.45in
\evensidemargin=0in
\oddsidemargin=0in
\textwidth=6.5in
\textheight=9.0in
\headsep=0.25in

\title{ \bf CNN Results: 20-35 Crossings }
\author{ Jake Weatherhead }
\date{\today}

\begin{document}
\maketitle	

\noindent
The same CNN and ViT used in the initial study were fine-tuned on a new dataset $\mathcal{D}$, where:\\

\[
\begin{gathered}
\mathcal{D} \; = \; \mathcal{D}_{\mathrm{train}} \; \cup \; \mathcal{D}_{\mathrm{val}} \; \cup \; \mathcal{D}_{\mathrm{test}},  \\[0.5em]
|\mathcal{D}| \; = \; 560,000 \text{ diagrams}, \\[0.5em]
|\mathcal{D}_{\mathrm{train}}| \; = \; 448{,}000 \text{ diagrams}, \\[0.5em]
|\mathcal{D}_{\mathrm{val}}| \; = \; |\mathcal{D}_{\mathrm{test}}| \; = \; 56{,}000 \text{ diagrams}.
\end{gathered}
\]\\


\noindent
All splits contained knots with $n$ crossings where $n \in N$ and $N=\{20, \,21, \dots , \,35\}$. Each crossing count $n$ contributed
equally to a split. In each split, there were an equal number of unknots and non-trivial knots with $n$ crossings.

\section*{CNN Results}

% Here, add the test accuracy by crossing count.

\subsection*{CNN TP: True Unknots Predicted as Unknots}
\subsection*{CNN FN: True Unknots Predicted as Knots}
\subsection*{CNN TN: True Knots Predicted as Knots}
\subsection*{CNN FP: True Knots Predicted as Unknots}

\end{document}