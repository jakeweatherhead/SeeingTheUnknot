\documentclass[11pt]{article}

\usepackage{sectsty}
\usepackage{graphicx}
\usepackage{amsmath}
\usepackage{float}

% Margins
\topmargin=-0.55in
\evensidemargin=0in
\oddsidemargin=0in
\textwidth=6.5in
\textheight=9.0in
\headsep=0.25in

\title{ \bf Results: [20, 35] Crossings }
\author{ Jake Weatherhead }
\date{\today}

\begin{document}
\maketitle	

\noindent
The same CNN and ViT used in the initial study were fine-tuned on a new dataset $\mathcal{D}$, where:\\

\[
\begin{gathered}
\mathcal{D} \; = \; \mathcal{D}_{\mathrm{train}} \; \cup \; \mathcal{D}_{\mathrm{val}} \; \cup \; \mathcal{D}_{\mathrm{test}}  \\[0.5em]
|\mathcal{D}| \; = \; 560,000 \text{ diagrams} \\[0.5em]
|\mathcal{D}_{\mathrm{train}}| \; = \; 448{,}000 \text{ diagrams} \\[0.5em]
|\mathcal{D}_{\mathrm{val}}| \; = \; |\mathcal{D}_{\mathrm{test}}| \; = \; 56{,}000 \text{ diagrams}.
\end{gathered}
\]\\


\noindent
Each split in $\mathcal{D}$ included knots with $n$ crossings, where $n \in \{20, \,21, \, \ldots \,, \,35\}$.
For every $n$, the number of unknot and non-trivial knot diagrams per split was identical.

\begin{figure}[H]
   \centering
   \includegraphics[width=1\textwidth]{tikz/NOV-MON-10/example-diagrams.pdf}
   \caption{35 crossing unknots (top row) and non-trivial knots (bottom row) in $\mathcal{D}_{\mathrm{test}}$.}
   \label{fig:examples-nov-mon}
\end{figure}

\newpage
\section*{CNN Results}

% Here, add the test accuracy by crossing count.

\subsection*{CNN TP: True Unknots Predicted as Unknots}
\subsection*{CNN FN: True Unknots Predicted as Knots}
\subsection*{CNN TN: True Knots Predicted as Knots}
\subsection*{CNN FP: True Knots Predicted as Unknots}

\newpage
\section*{ViT Results}

% Here, add the test accuracy by crossing count.

\subsection*{ViT TP: True Unknots Predicted as Unknots}
\subsection*{ViT FN: True Unknots Predicted as Knots}
\subsection*{ViT TN: True Knots Predicted as Knots}
\subsection*{ViT FP: True Knots Predicted as Unknots}

\end{document}