\documentclass[11pt]{article}

\usepackage{sectsty}
\usepackage{graphicx}
\usepackage{amsmath}

% Margins
\topmargin=-0.45in
\evensidemargin=0in
\oddsidemargin=0in
\textwidth=6.5in
\textheight=9.0in
\headsep=0.25in

\title{ \bf CNN Results: 20-35 Crossings }
\author{ Jake Weatherhead }
\date{\today}

\begin{document}
\maketitle	

\noindent
The 88M-parameter $\textsc{CNN}$ was fine-tuned on $\mathcal{D}_{\text{train}}$,
validated on $\mathcal{D}_{\text{val}}$, and tested on $\mathcal{D}_{\text{test}}$, where:\\

\[
\begin{aligned}
&|\mathcal{D}_{\text{train}}| \; = \; 448{,}000 \text{ diagrams}, \\[0.5em]
&|\mathcal{D}_{\text{val}}| \; = \; |\mathcal{D}_{\text{test}}| \; = \; 56{,}000 \text{ diagrams}, \\[0.5em]
&\mathcal{D} \; = \; \mathcal{D}_{\text{train}} \; \cup \; \mathcal{D}_{\text{val}} \; \cup \; \mathcal{D}_{\text{test}}.
\end{aligned}
\]\\

\noindent
For each split in $\mathcal{D}$, each knot had $n \in N$ crossings where $N=[20,35]$. For all splits, and all
values of $n$, there was a exact parity of knots and unknots and there were an equal number of diagrams for each value of $n \in N$.

\section*{Results}

% Here, add the test accuracy by crossing count.

\subsection*{TP: True Knots Predicted as Knots}
\subsection*{FN: True Knots Predicted as Knots}
\subsection*{TN: True Knots Predicted as Knots}
\subsection*{FP: True Knots Predicted as Unnots}

\end{document}