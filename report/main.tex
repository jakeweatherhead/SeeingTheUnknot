\documentclass[11pt]{article}
\usepackage[a4paper, top=0.15in, bottom=0.9in, left=0.6in, right=0.6in]{geometry}

\usepackage{sectsty}
\usepackage{graphicx}
\usepackage{amsmath}
\usepackage{float}

\title{ \bf Results: [20, 35] Crossings }
\author{ Jake Weatherhead }
\date{\today}

\begin{document}
\maketitle	

\noindent
The CNN (88M params) and ViT (86M params) were fine-tuned on a new dataset $\mathcal{D}$ comprising unknot $\mathcal{U}$ and non-trivial knot
$\mathcal{K}$ diagrams partitioned into three disjoint data splits $S := \{$train, val, test$\}$, where:\\

\[
\begin{gathered}
\mathcal{D} \; = \; \biguplus_{s \in S} \mathcal{D}_s \\[0.5em]
|\mathcal{D}| \; = \; 560{,}000 \,\text{ diagrams, } \\[0.5em]
|\mathcal{D}_{\mathrm{train}}| \; = \; 0.8 \cdot |\mathcal{D}| \; = \; 448{,}000 \text{ diagrams,} \\[0.5em]
|\mathcal{D}_{\mathrm{val}}| \; = \; |\mathcal{D}_{\mathrm{test}}| \; = \; 0.1 \cdot |\mathcal{D}| \; = \; 56{,}000 \text{ diagrams}, \\[0.5em]
\text{where } \; \forall s \in S, \;\, s \in \; \mathcal{U}_s \, \cup \, \mathcal{K}_s \; \text{ and } \; |\mathcal{U}_s| = |\mathcal{K}_s|. \\[0.5em]
\end{gathered}
\]\\[0.5em]

\noindent
Each split contained knot diagrams for every crossing count $n \in N := \{20, \, 21, \, \ldots \,, \,35 \}$.
For all distinct $p, q \in N$, each split contained an equal number of $p$ and $q$-crossing unknot and non-trivial knot diagrams.

\section*{Learning Curves}
\vspace{-1.3em}
\begin{figure}[H]
   \centering
   \includegraphics[width=0.59\textwidth]{/home/jake/Personal/SeeingTheUnknot/report/tikz/NOV-MON-10/cnn-lc.png}
   \includegraphics[width=0.59\textwidth]{/home/jake/Personal/SeeingTheUnknot/report/tikz/NOV-MON-10/vit-lc.png}
\end{figure}
\vspace{-2em}

\newpage
\section*{CNN* Saliency Maps}

Saliency maps for knot diagrams in $\mathcal{D}_{\mathrm{test}}$ for the best-performing CNN (CNN*).

% Here, add the test accuracy by crossing count.

\subsection*{CNN* True Positives: True Unknots Predicted as Unknots}



\subsection*{CNN* False Negatives: True Unknots Predicted as Knots}
\vspace{-1em}
\begin{figure}[H]
   \centering
   \includegraphics[width=1\textwidth]{tikz/NOV-MON-10/cnn/FN/cnn-fn.pdf}
   \label{fig:cnn-fn}
\end{figure}
\vspace{-1em}

\subsection*{CNN* True Negatives: True Knots Predicted as Knots}
\subsection*{CNN* False Positives: True Knots Predicted as Unknots}
\vspace{-1em}
\begin{figure}[H]
   \centering
   \includegraphics[width=1\textwidth]{tikz/NOV-MON-10/cnn/FP/cnn-fp.pdf}
   \label{fig:cnn-fp}
\end{figure}
\vspace{-1em}

\newpage
\section*{ViT* Saliency Maps}

Saliency maps for knot diagrams in $\mathcal{D}_{\mathrm{test}}$ for the best-performing ViT (ViT*).
% Here, add the test accuracy by crossing count.

\subsection*{ViT TP: True Unknots Predicted as Unknots}
\subsection*{ViT FN: True Unknots Predicted as Knots}
\subsection*{ViT TN: True Knots Predicted as Knots}
\subsection*{ViT FP: True Knots Predicted as Unknots}

\end{document}